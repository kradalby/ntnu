\documentclass[a4paper, 10pt]{article}
\usepackage[utf8x]{inputenc}
\usepackage[norsk]{babel}
\usepackage{natbib}
\usepackage{graphicx}
\usepackage[T1]{fontenc}
\usepackage{amsmath}
\usepackage{mathtools}

\title{TDT4145 - Øving 1}
\author{Kristoffer Dalby}
\date{}


\begin{document}

\maketitle

\thispagestyle{empty}
\newpage
\pagenumbering{arabic}
\setcounter{page}{1}

\section*{Oppgave 1}
\subsection*{a)}
Data er et sett med verdier og informasjon. Man kan se på data som deler av helhetlig informasjon.\\\\
En database er en samling data som er satt i struktur som gjør den enkel å håndtere basert på hvilket formål den skal brukes til.\\\\
Et databasesystem er et samlet begrep som beskriver kombinasjonen av en database, et database håndteringssystem(PostgreSQL) og en data model.\\\\
Sammenhengen mellom systemene er veldig enkel, for at data skal være nyttig må den være strukturert og derfor putter vi den i en database. For å håndtere databasen enkelt og effektivt bruker vi et databasesystem for dette.

\subsection*{b)}
I et tradisjonelt filsystem vil alle bruker nødvendigvis måtte lage alle funksjoner de har behov for selv. Ettersom en database er svært nøye bygget på en struktur til et prosjekt vil man kunne implementere den en gang, istede for å måtte reimplementere den en haug med ganger. Et viktig punkt å nevne er at man med en database også skiller programmet og dataen.

\subsection*{c)}
Et databasesystem har funksjonalitet som de fleste filsystemer mangler. Dette er for eksempel håndtering av samtidige lesing og skriving til samme ressurs. En database vil også la deg gjøre "fast indexing" som i hovedsak lar deg indeksere data på en variable attribut eller data. Det er også i databasesystemer implementert en solid mengde integritets kontroller og feil rettinger for at dataen skal kunne være inntakt og håndteres riktig.




\end{document}
