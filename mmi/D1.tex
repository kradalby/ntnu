\documentclass[a4paper, 10pt]{article}
\usepackage[utf8x]{inputenc}
\usepackage[norsk]{babel}
\usepackage{natbib}
\usepackage{graphicx}
\usepackage[T1]{fontenc}
\usepackage{amsmath}
\usepackage{mathtools}

\title{MMI D1}
\author{Kristoffer Dalby}
\date{}


\begin{document}

\maketitle

\thispagestyle{empty}
\newpage
\pagenumbering{arabic}
\setcounter{page}{1}

\section*{Nettsider}
Jeg har i denne oppgaven valgt å analysere One Call (onecall.no) og Telenor (telenor.no).

\section*{Målgruppe}
Målgruppen for nettstedene er kunder av de aktuelle telefoni selvskapene og potensielle mobilkunder. Dette er nordmenn i alle aldere som er aktuelle som brukere av mobiltelefoner eller mobilt bredbånd. I denne oppgaven har jeg valgt å jobbe utifra et sett med målgrupper beskrevet under. Jeg har også valgt å se bort ifra at Telenor tilbyr bedriftstjenester, da One Call ikke gjør dette.\\

\subsection*{Persona}
\subsubsection*{Studenten}
Thor Martin er 22 år og student, han bor i Trondheim, har kjæreste i Sandefjord og aktiv i mange verv. Han er derfor avhengig av mobiltelefon for kommunisere med alle, enklest mulig. Han er en stor bruker av både mobil og mobilt internett.

\subsubsection*{Familiefaren}
Ander er 48 år og bor i Oslo, han har fast jobb og familie. Han driver eget selvskap sammen sin kone og spiller golf på fritiden.

\subsubsection*{Pensjonisten}
Karin er 77 år, bor i Sandefjord, hun bruker mobiltelefon mye etter hun har sagt opp fasttelefonen. Hun snakker med sine venninner og familie. 


\newpage
\section*{One Call}
\subsection*{Visuell fremstilling}
One Call sin nettside har et oversiktlig rent design med klare linjer. Det har ikke overdreven fargebruk som gjør at det er behaglig å se på. Nettsiden har en visuell fremstilling som er relativ nøytral og gjør den egnet til å dekke alle sine målgrupper. Fargevalget er basert på selvskapets egne farger som er en del av deres image.\\


\subsection*{Gestalt prinsippene}

\begin{description}
  \item[Gruppering/Nærhet] \hfill \\
    Nettsiden har et lagbasert design, hvor hvert lag har undergrupper samler informasjon som er relatert.
  \item[Linje/Kontinuitet] \hfill \\
    One call sin nettside er som nevnt lagbasert, mellom hvert lag er det et skille før det neste begynner, på grunn av kraftig farge endring kan man se dette som klare linjer. 
  \item[Mental komplettering] \hfill \\
    Det er ingen tydlige tegn på mental komplettering.  
  \item[Likhet i form] \hfill \\
    Nettsiden følger en lagdesignet gjennom alle sidene, men på flere av sidene er det midterste laget mye større en på forsiden. Det er på disse sidene mer informasjon som er plassert i tabeller og bokser, hovedsaklig firkant baserte former.
  \item[Likhet i farge] \hfill \\
    Nettsiden har et generelt preg av fargen rød og flere nyanser mellom hvit og grå.
  \item[Forgrunn/Bakgrunn] \hfill \\
    Det er ingen klare skiller som utgjør om noe hører til forgrunn eller bakgrunn.
\end{description}

\subsubsection*{Jacob Nielsens 10 punkter}

\begin{description}
  \item[1. Visibility of system status] \hfill \\
    Det er ikke mye informasjon som indikerer hvilke side man er på, det er en liten pil ned fra hovedmenyen som indikerer hvilke side man er på.
  \item[2. Match between system and real world] \hfill \\
    Det er på forsiden brukt et bilde for å promotere som gjennspeiler virkeligheten. Det blir brukt et bilde av en ung dame i mobiltelefonen.
  \item[3. User control and freedom] \hfill \\
    På nettstedet er det meste av informasjon tilgjenglig som omhandler produktene til OneCall, men det er en del av nettsiden som er forbeholdt kunder da dette omhandler å endre deres abonnomenter.
  \item[4. Consistency and standards] \hfill \\
    One Call har alle vært konsise og passet på at de bruker enkle fagutrykk som går igjen.
  \item[5. Error prevention] \hfill \\
    Jeg har i min test ikke funnet noen feil eller fått opp noen feilmeldinger. Jeg vil derfor påstå at det er godt håndtert.
  \item[6. Recognintion reather than recall] \hfill \\
    Siden er veldig konsis gjennom de forskjellige sidene under nettstedet. Det er samtidig mulig å se likhetstrekk mellom Telenor og One Call sin side, dette gjør at en kunde vil enklere kunne navigere begge.
  \item[7. Flexibility and effiency of use] \hfill \\
    Nettstedet har et godt design hvor det meste av informasjon er tilgjenglig ved få klikk fra forsiden. Det er også et søkefelt som kan hjelpe med å øke effektiviteten.
  \item[8. Aestetic and minimalist design] \hfill \\
    Nettsiden har en god balanse når det kommer til mengden informasjon per side. Den er også estetisk pen om man liker rødfargen som er valgt.
  \item[9. Help users recognize, diagnose, and recover from errors] \hfill \\
    Det har i hovedsak ikke forekommet noen feil mens jeg testet siden, derfor har jeg ikke noe å utdype på dette punktet.
  \item[10. Help and documentation] \hfill \\
    Det er ingen åpenbar dokumentasjon på hvordan man bruker nettsiden, og under Kundeservice på forsiden er det ingenting som forklarer hvordan man kan bruke siden. Det er dog laget en side for å forklare hvordan man bestiller abonnement.
\end{description}

\subsubsection*{Schneiderman's 8 punkter}

\begin{description}
  \item[1. Strive for Consistency] \hfill \\
    One Call sin nettside har et konsist design gjennom alle sidene.
  \item[2. Enable frequent users to use shortcuts] \hfill \\
    Det er ikke vist frem noen implementasjon av hurtigtaster, men de har nederst på forsiden en litt mer detaljert meny hvor det er linker direkte til endel undersider med mer spesifikk informasjon.
  \item[3. Offer informative feedback] \hfill \\
    One call sin nettside har mye feedback når man holder pekeren over linker og meny objekter.
  \item[4. Design dialogs to yield closure] \hfill \\
    Hovedsiden har fokuset, og på denne siden handler det om at du skal bli kunde eller finne frem til informasjon som er relevant for en eksisterende kunde eller en ny kunde. Nettsiden har flere veivisere som hjelper deg fra start til slutt.
  \item[5. Offer simple error handling] \hfill \\
    Se Punkt 5 i Jacob Nielsens liste.
  \item[6. Permit easy reversal of actions] \hfill \\
    Om det oppstår en feil kan alltid brukeren laste sidene på nytt.  
  \item[7. Support internal locus of control] \hfill \\
    Brukeren er sjefen, han eller hun bestemmer hvor det skal navigeres, og hvordan siden skal brukes.
  \item[8. Reduce short term memory] \hfill \\
    Nettsiden er så enkel og oversiktlig bygget opp at det burde ikke være noe problem å navigere seg til korrekt sted fra en gitt undernettside på nettstedet.
\end{description}


\subsection*{Språk}
One Call sin nettside er preget av at alt skal være enkelt og alt skal skje på nett. De har valgt et språk som er enkelt og forstå å bruker kun de faguttrykkene som er nødvendige (som for eksempel SIM-kort).

\subsection*{Oppbyggning}
One Call sitt nettsted er bygget opp slik at alle hyperlinker har en effekt når man har pekeren over dem. Dette gir en god indikasjon på hvor man er på nettsiden. Menyen er bygget opp med hovedkategorier og er veldig oversiktlig. Det er også en indikator som forteller deg hvilke av "Hovedsidene" man er på. Det meste av informasjon på siden har et selvforklarende bruksmønster ettersom det i seg selv er en tekst man kan trykke på. Det er også bygget veldig konsistent slik at en bruker skal kunne kjenne seg igjen uansett hvor de er, samtidig som man kan nå alle hovedsidene fra enhver annen side.


\newpage
\section*{Telenor}
\subsection*{Visuell fremstilling}
Telenor sin nettside er stor grad bygget på samme prinsippet som Onecall, den har et litt annet oppsett og presentasjon av informasjonen. Nettsiden har et oversiktlig oppsett av lenker videre til mer informasjon og fargevalget er rolig og avslappende.\\
Telenor har imotsettning til OneCall valgt å ha den promoterende banneren helt øverste.


\subsection*{Gestalt prinsippene}

\begin{description}
  \item[Gruppering/Nærhet] \hfill \\
    Nettsiden har et lagbasert design. Nettsiden har som lag nummer to, en gruppe med flere undergrupper som representerer menyen til nettsiden.
  \item[Linje/Kontinuitet] \hfill \\
    Telenor sin nettside har en litt dårligere kontiniutet en OneCall har, mye av Telenor sin nettside bygger rundt nettbutikken deres, som har et mildt endrett oppsett når det kommer til linjer og plasseringer av objekter.
  \item[Mental komplettering] \hfill \\
    Det er ingen tydlige tegn på mental komplettering.  
  \item[Likhet i form] \hfill \\
    Som nevnt over er siden delt opp i nettbutikk og informasjonsider. Det er en ulikhet mellom disse større hovedgruppene, men internt i dem er den lik.
  \item[Likhet i farge] \hfill \\
    Nettsiden har et generelt preg av fargene sort og blått, samt hvit som bakgrunn.
  \item[Forgrunn/Bakgrunn] \hfill \\
    Det er ingen klare skiller som utgjør om noe hører til forgrunn eller bakgrunn.
\end{description}

\subsubsection*{Jacob Nielsens 10 punkter}

\begin{description}
  \item[1. Visibility of system status] \hfill \\
    Telenor har en tekstlig informasjon øverst på nettsiden som indikerer hvor du er i systemet. Den kan være uoversiktlig for brukere som ikke har erfaring med navigering på internett
  \item[2. Match between system and real world] \hfill \\
    Forsiden har til tider preg av det norske landskapet.
  \item[3. User control and freedom] \hfill \\
    På nettstedet er det meste av informasjon tilgjenglig som omhandler produktene til Telenor, men det er en del av nettsiden som er forbeholdt kunder da dette omhandler å endre deres abonnomenter.
  \item[4. Consistency and standards] \hfill \\
    Telenor bruker enkle fagutrykk slik at de fleste brukere kan finne fram.
  \item[5. Error prevention] \hfill \\
    Jeg har i min test ikke funnet noen feil eller fått opp noen feilmeldinger. Jeg vil derfor påstå at det er godt håndtert.
  \item[6. Recognintion reather than recall] \hfill \\
    Nettstedet til telenor er delt i nettbutikk og informasjon, noe som ikke er sa konsist. Det som er greit å ta med seg er at informasjonsbiten av Telenor sin side har en relativt lik oppbygging som One Call.
  \item[7. Flexibility and effiency of use] \hfill \\
    Nettsiden har et design som fokuserer på at du skal kunne nå de viktigste funksjonene fra forsiden.
  \item[8. Aestetic and minimalist design] \hfill \\
    Det er mye informasjon på forsiden, noe som gjør at den ikke kan kalles minimalistisk. Den er estetisk pen, og har et mye roligere fargevalg enn OneCall.
  \item[9. Help users recognize, diagnose, and recover from errors] \hfill \\
    Det har i hovedsak ikke forekommet noen feil mens jeg testet siden, derfor har jeg ikke noe å utdype på dette punktet.
  \item[10. Help and documentation] \hfill \\
    Det er heller ikke på telenor sin nettside noen åpenbar tilgjenglig dokumentasjon for selve nettstedet. Men de har fokusert på å få frem kundeservice som kan hjelpe til med problemer.
\end{description}

\subsubsection*{Schneiderman's 8 punkter}

\begin{description}
  \item[1. Strive for Consistency] \hfill \\
    Telenor har et konsist fargevalg og står på rette linjer. Men de har et litt annerledes design for nettbutikken sin.
  \item[2. Enable frequent users to use shortcuts] \hfill \\
    Nettsiden er bygget så viktige funksjoner kommer fram på forsiden og gjør det raskt og enkelt å finne frem.
  \item[3. Offer informative feedback] \hfill \\
    Telenor sin nettside har mye feedback når man holder pekeren over linker og meny objekter. Det er dog mindre oversiktlig hvilken side man er på.
  \item[4. Design dialogs to yield closure] \hfill \\
    Det er et stort fokus for å få deg til nettbutikken til telenor, hvor du kan handle abonnementer og telefoner. Derifra har de veivisere for å hjelpe deg fra start til slutt.
  \item[5. Offer simple error handling] \hfill \\
    Se Punkt 5 i Jacob Nielsens liste.
  \item[6. Permit easy reversal of actions] \hfill \\
    Om det oppstår en feil kan alltid brukeren laste sidene på nytt.  
  \item[7. Support internal locus of control] \hfill \\
    Brukeren har full kontroll over hvordan siden skal navigeres, men det er mye oppfordninger som kan prøve å få deg inn på spesifikke sider.
  \item[8. Reduce short term memory] \hfill \\
    Nettsiden har mye informasjon på forsiden, noe som gjør at man kanskje må huske litt mer på hvor ting er så man slipper å lese over alle menyer hver gang man besøker nettsiden.
\end{description}

\subsection*{Språk}
Telenor sin nettside har en litt annen språkprofil en det OneCall har. I hovedsak har vi også fokusert på å gjøre det enkelt for brukeren. Telenor har dog en fokus på at du kan kontakte oss hvis du trenger hjelp, istede for at det meste skal gjøres på nettsiden deres. Telenor kan kanskje også sies å ha et litt mer voksent språk.

\subsection*{Oppbyggning}
Nettsiden til Telenor har istede for å ha en hovedmeny i toppen valgt å heller ha et sett med mer detaljerte menyer litt lenger ned på nettsiden. Her har du større kategorier som Mobil og Bredbånd. Hver av disse undermenyene har listet opp 5 linker til tjenester det er sannsynlig at du er interessert i å søke. Alle disse linkene har en markerings effekt som gjør det tydlig at de er mulig å trykke på. Det er verdt å merke seg at denne menyen ikke er konsistent plassert på samme plass, men blir gjemt under en dropdown meny merket Alt innhold. Det er derfor ikke helt klart hvor man kan gå og hvor man er fra en gitt side på telenor sin nettside.




\subsection*{Brukervennlighet}
One Call har en salgsmodell om at de er et mobilselvskap på nettet, så de har et veldig stort fokus på at man skal kunne gjøre alt man trenger å gjøre angående et mobilabonnement på nettet. Nettsiden er derfor laget svært brukervennlig slik at kundene skal klare å gjøre dette selv. \\

Telenor på den andre siden har en side som har mye informasjon på hver side og kan være vanskligere å få oversikt over. Telenor har gjort det enklere å få kontakt med kundeservice på telefon, noe som for eldre brukere kan være en god ide.

\section*{Hensikt}
\subsection*{Scenario 1 - Bli kunde}
Et typisk scenario for begge nettsidene vil være å bli kunde. Begge nettsidene er bygget slik at dette er en veiviser hvor du går gjennom flere steg for å oppnå dette.\\

Thor Martin er misfornøyd med sin mobilleverandør idag og ønsker å bytte. Han navigerer derfor til Onecall sin nettside. Har leser han rundt på siden og finner ut at det er enkelt å starte veiviseren som hjelper han å bytte og samtidig overføre nummeret hans til den nye leverandøren. Thor Martin er fornøyd med at de hjelper han å flytte alt. Thor Martin er glad.

\subsection*{Scenario 2 - Oppdatere informasjon}
Det er også mulig for eksisterende brukere av nettsiden å logge inn for å oppdatere informasjonen sin.\\

Annine har idag Telenor som sin leverandør og dette er hun strålende fornøyd med. Men hun har nylig flyttet og er nødt til å oppdatere fakturaaddressen sin. Hun går derfor inn på Telenor sin nettside hvor hun raskt og enkelt logger inn på Min side. Her oppdaterer hun addressen og velger den som ny faktura addresse.

\subsection*{Scenario 3 - Kontakt med kundeservice}
Det vil alltid være folk som heller ønsker å snakke med noen, istede for å navigere seg rundt på en nettside til informasjon. Dette er gjerne et scenario for den eldre garde.

Karin har spørsmål om fakturaen hun akkuratt har mottatt, hun ønsker derfor å snakke med en person som jobber hos telenor og har tilgang til å gi henne korrekt informasjon. Hun navigerer derfor til telenor.no med sin datamaskin. På forsiden finner hun en informasjons blokk hvor det står Ring meg. Hun trykker på linken og blir videreført til en ny side hvor hun kan taste inn sitt telefonnummer så telenor ringer henne. Hun gjør dette og er strålende fornøyd med at hun ikke bør stå i kø for å få tak i en kundebehandler.

\section*{Sammenligning}
Nettsidene skiller seg i hovedsak på fargevalg og profil valg. Telenor har valgt et mørkt design som gir siden et ekstra preg av proffesjonalitet. Hvor Onecall heller har valgt et litt mer ungt og hipt design. Telenor har også en avdeling for bedrift og Onecall ikke har dette. De har heller dette som et eget selvskap, Network Norway. Onecall bruker også mye  av sin nettside på å vise frem at de har norges mest fornøyde kunder og andre fordeler de mener de kan tilby iforhold til andre. Telenor holder seg her helt nøytrale. 

\end{document}
