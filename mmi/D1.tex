\documentclass[a4paper, 10pt]{article}
\usepackage[utf8x]{inputenc}
\usepackage[norsk]{babel}
\usepackage{natbib}
\usepackage{graphicx}
\usepackage[T1]{fontenc}
\usepackage{amsmath}
\usepackage{mathtools}

\title{MMI D1}
\author{Kristoffer Dalby}
\date{}


\begin{document}

\maketitle

\thispagestyle{empty}
\newpage
\pagenumbering{arabic}
\setcounter{page}{1}

\section*{Nettsider}
Jeg har i denne oppgaven valgt å analysere One Call (onecall.no) og Telenor (telenor.no).

\section*{Målgruppe}
Målgruppen for nettstedene er kunder av de aktuelle telefoni selvskapene og potensielle mobilkunder. Dette er nordmenn i alle aldere som er aktuelle som brukere av mobiltelefoner eller mobilt bredbånd. I denne oppgaven har jeg valgt å jobbe utifra et sett med målgrupper beskrevet under. Jeg har også valgt å se bort ifra at Telenor tilbyr bedriftstjenester, da One Call ikke gjør dette.\\

\subsection*{Persona}
\subsubsection*{Studenten}
Thor Martin er 22 år og student, han bor i Trondheim, har kjæreste i Sandefjord og aktiv i mange verv. Han er derfor avhengig av mobiltelefon for kommunisere med alle, enklest mulig. Han er en stor bruker av både mobil og mobilt internett.\\

\subsubsection*{Familiefaren}
Ander er 48 år og bor i Oslo, han har fast jobb og familie. Han driver eget selvskap sammen sin kone og spiller golf på fritiden.

\subsubsection*{Pensjonisten}
Karin er 77 år, bor i Sandefjord, hun bruker mobiltelefon mye etter hun har sagt opp fasttelefonen. Hun snakker med sine venninner og familie. 

\section*{One Call}
\subsection*{Visuell fremstilling}
One Call sin nettside har et oversiktlig rent design med klare linjer. Det har ikke overdreven fargebruk som gjør at det er behaglig å se på. Nettsiden har en visuell fremstilling som er relativ nøytral og gjør den egnet til å dekke alle sine målgrupper. Fargevalget er basert på selvskapets egne farger som er en del av deres image.\\


\subsection{Gestalt prinsippene}

\begin{description}
  \item[Gruppering/Nærhet] \hfill \\
  Elementer som ligner på hverandre er samlet i grupper som produkter og support. Videre er det egne grupper for produktene enten i hovedmenyen som Apple har eller under produkt bildene som samsung.    
  \item[Linje/Kontinuitet] \hfill \\
  Begge nettsiden har klare linjer og kontinuitet i formen, og er preget av firkanter/tiles som utgangspunkt for oppdelingen.
  \item[Mental komplettering] \hfill \\
    Finner ingen tegn til mental komplettering
   \item[Likhet i form] \hfill \\
    Nettsidene er preget av likhet i formen og følger en standard gjennom siden. Samsung bruker litt flere former enn hva Apple gjør og har litt variasjon i formen.  
  \item[Likhet i farge] \hfill \\
    Begge sidene er preget av en hovedfarge med kontrast farger for å fremheve enkelte deler, Apple bruker lyse farger mens samsung har litt mørkere. 
    \newpage
  \item[Forgrunn/Bakgrunn + reversering] \hfill \\
    På begge nettsidene er hovedmeny linjen i forgrunnen. På samsung sin nettside er også produkt menyen i forgrunnen av resten av siden og annonsene er lagt i bakgrunnen av siden. 
\end{description}

\subsubsection{Jacob Nielsens 10 punkter}

\begin{description}
  \item[1. Visibility of system status] \hfill \\
    Det er begrenset med status på nettsidene, men vi har derimot indikatorer som viser hvilken del av siden man er inne på, sånn at man alltid har oversikt over hvilken del man ser på.
  \item[2. Match between system and real world] \hfill \\
     Det er brukt en naturlig skrift som kan relateres til håndskrift, videre er også fargekombinasjonene naturlige i naturen. Dette gjelder begge nettsidene. 
  \item[3. User control and freedom] \hfill \\
     Brukeren kan bevege seg fritt på nettsidene og det er ingen begrensning på hva man kan trykke på. 
  \item[4. Consistency and standards] \hfill \\
    Det ser ut til at standarder og kjente ord er brukt på begge sidene, har ikke funnet noen inkonsistens på noen av sidene.      
  \item[5. Error prevention] \hfill \\
    Får ingen feil på nettsidene og det ser ut til at error prevention fungerer godt. 
  \item[6. Recognintion reather than recall] \hfill \\
  Nettsiden består av elementer som er lette å kjenne igjen og man trenger ikke huske enkelte deler for å navigere på siden. 
  \item[7. Flexibility and effiency of use] \hfill \\
     Det finnes ingen egen funksjonalitet for ekspert brukere på begge sidene men de har et søkefelt som kan øke effektivteten hvis man vet hva man ønsker å finne. 
  \item[8. Aestetic and minimalist design] \hfill \\
    Her vil jeg si at Apple er flinkere på det minimalistiske designet, Samsung tar med en del informasjon fra Facebook og andre parter som nødvendigvis ikke er av betydning for brukeren. 
  \item[9. Help users recognize, diagnose, and recover from errors] \hfill \\
      Opplevde ingen problemer med siden og fikk ingen feil som gjorde at jeg fikk testet dette punktet. 
  \item[10. Help and documentation] \hfill \\
    Både Apple og Samsung har support knapper på hovedmenyen som mest sannsynlig ledere videre til hjelp og bruk av nettsiden. 
\end{description}

\subsubsection{Schneiderman's 8 punkter}

\begin{description}
  \item[1. Strive for Consistency] \hfill \\
    Begge sidene er preget av konsistens i designet, det er ingen tydlelige forskjellere og endringen på hvordan nettsidene ser ut.  
  \item[2. Enable frequent users to use shortcuts] \hfill \\
    Ser ingen snarveier eller tastaturkommandoer som leder direkte til enkelte produkter, men du har menyer og annonser som kan klikkes på og leder videre til en underside. 
  \item[3. Offer informative feedback] \hfill \\
    Apple sin side gir tilbakemelding på hvilken del av menyen pekeren er på, Samsung gjør det samme men går også et skritt videre ved å vise undermenyer når man holder pekeren over.  
  \item[4. Design dialogs to yield closure] \hfill \\
    Fokuset er på hovedsiden og det er dermed ikke mulig å gjennomføre en serie med handlinger som leder til en slutt, som et kjøp. 
  \item[5. Offer simple error handling] \hfill \\
    Har ikke mulighet til å vite hvordan nettsidene behandler feil
  \item[6. Permit easy reversal of actions] \hfill \\
    Om det oppstår en feil kan alltid brukeren laste sidene på nytt.  
  \item[7. Support internal locus of control] \hfill \\
    Brukerne er i full kontroler over navigeringen av sidene og det blir ikke bestemt av siden hvor de skal trykke. 
  \item[8. Reduce short term memory] \hfill \\
    Begge nettsidene er enkle i bruk og gir informasjon etterveis som man navigerer i dem, de krever ikke at brukerne husker delere av navigeringen eller informasjon som blir gitt for å gå videre.  
\end{description}

Begge nettsidene er bygget opp lagvis, de har forskjellig informasjon på hvert lag. Det øverste laget innholder en overådet meny for større viktigere valg og logoen til bedriftene. Det neste laget er en større grafikk som fanger blikket med aktuelle tilbud. Under dette kommer en meny med oversikt over alle produktene selvskapene tilbyr. Det lagbaserte designet er et gesaltprinsipp. Et menneske deler opp informasjonen og det er enkelt å kontinuerlig finne informasjon.\\

Ja, såvidt jeg kan se følges de fleste punktene i guidlines for design.


\subsection*{Hensikt}
\subsubsection*{Scenario 1 - Bli kunde}
Et typisk scenario for begge nettsidene vil være å bli kunde. Begge nettsidene er bygget slik at dette er en veiviser hvor du går gjennom flere steg for å oppnå dette.\\

Thor Martin er misfornøyd med sin mobilleverandør idag og ønsker å bytte. Han navigerer derfor til Onecall sin nettside. Har leser han rundt på siden og finner ut at det er enkelt å starte veiviseren som hjelper han å bytte og samtidig overføre nummeret hans til den nye leverandøren. Thor Martin er fornøyd med at de hjelper han å flytte alt. Thor Martin er glad.

\subsubsection*{Scenario 2 - Oppdatere informasjon}
Det er også mulig for eksisterende brukere av nettsiden å logge inn for å oppdatere informasjonen sin.\\

Annine har idag Telenor som sin leverandør og dette er hun strålende fornøyd med. Men hun har nylig flyttet og er nødt til å oppdatere fakturaaddressen sin. Hun går derfor inn på Telenor sin nettside hvor hun raskt og enkelt logger inn på Min side. Her oppdaterer hun addressen og velger den som ny faktura addresse.

\subsection*{Språk}
Valget av ord er enkelt og beskrivende. Det er konkret valgt ord som gjør det enkelt for alle å ha en oversikt over hva de finner av informasjon på nettsidene. Det er ingen eller lite store ord eller fremmed ord på siden, som gjør alle aldre og ikke-tekniske personer fult egnet til å navigere siden.

\subsection*{Oppbyggning}
Nettsidene er bygget opp veldig likt, og alt av hyperlinker er veldig oversiktlig å holde styr på. Menyen har en oversiktlig oppbyggning, det er ingen dropdown menyer som gjør ting uoversiktlig og man kan nå de fleste viktigie tjenester fra forsiden.

Normans

\subsection*{Brukervennlighet}
Begge nettsidene har en veldig brukervennlig og funksjonell nettside, jeg vil personlig si at one call sin virker snevert mer brukervennlig en Telenor sin nettside. Fargevalget på telenor er litt roligere med mørkere farger som gjør vennligheten på øynene litt roligere, men One Call har til gjengjeld gjort mye som kan være avansert simplere.\\

Begge nettstedene er enkelt for alle målgruppene å nå. Det meste av informasjon er bygget opp slik at man finner det enkelt og veiviserene tar deg gjennom ting med mange valg.

\section*{Telenor}
\subsection*{Visuell fremstilling}
One Call sin nettside har et oversiktlig rent design med klare linjer. Det har ikke overdreven fargebruk som gjør at det er behaglig å se på. Nettsiden har en visuell fremstilling som er relativ nøytral og gjør den egnet til å dekke alle sine målgrupper. Fargevalget er basert på selvskapets egne farger som er en del av deres image.\\

Telenor sin nettside er stor grad bygget på samme prinsippet som Onecall, den har et litt annet oppsett og presentasjon av informasjonen. Nettsiden har et oversiktlig oppsett av lenker videre til mer informasjon og fargevalget er rolig og avslappende.\\

Begge nettsidene er bygget opp lagvis, de har forskjellig informasjon på hvert lag. Det øverste laget innholder en overådet meny for større viktigere valg og logoen til bedriftene. Det neste laget er en større grafikk som fanger blikket med aktuelle tilbud. Under dette kommer en meny med oversikt over alle produktene selvskapene tilbyr. Det lagbaserte designet er et gesaltprinsipp. Et menneske deler opp informasjonen og det er enkelt å kontinuerlig finne informasjon.\\

Ja, såvidt jeg kan se følges de fleste punktene i guidlines for design.


\subsection*{Hensikt}
\subsubsection*{Scenario 1 - Bli kunde}
Et typisk scenario for begge nettsidene vil være å bli kunde. Begge nettsidene er bygget slik at dette er en veiviser hvor du går gjennom flere steg for å oppnå dette.\\

Thor Martin er misfornøyd med sin mobilleverandør idag og ønsker å bytte. Han navigerer derfor til Onecall sin nettside. Har leser han rundt på siden og finner ut at det er enkelt å starte veiviseren som hjelper han å bytte og samtidig overføre nummeret hans til den nye leverandøren. Thor Martin er fornøyd med at de hjelper han å flytte alt. Thor Martin er glad.

\subsubsection*{Scenario 2 - Oppdatere informasjon}
Det er også mulig for eksisterende brukere av nettsiden å logge inn for å oppdatere informasjonen sin.\\

Annine har idag Telenor som sin leverandør og dette er hun strålende fornøyd med. Men hun har nylig flyttet og er nødt til å oppdatere fakturaaddressen sin. Hun går derfor inn på Telenor sin nettside hvor hun raskt og enkelt logger inn på Min side. Her oppdaterer hun addressen og velger den som ny faktura addresse.

\subsection*{Språk}
Valget av ord er enkelt og beskrivende. Det er konkret valgt ord som gjør det enkelt for alle å ha en oversikt over hva de finner av informasjon på nettsidene. Det er ingen eller lite store ord eller fremmed ord på siden, som gjør alle aldre og ikke-tekniske personer fult egnet til å navigere siden.

\subsection*{Oppbyggning}
Nettsidene er bygget opp veldig likt, og alt av hyperlinker er veldig oversiktlig å holde styr på. Menyen har en oversiktlig oppbyggning, det er ingen dropdown menyer som gjør ting uoversiktlig og man kan nå de fleste viktigie tjenester fra forsiden.

\subsection*{Brukervennlighet}
Begge nettsidene har en veldig brukervennlig og funksjonell nettside, jeg vil personlig si at one call sin virker snevert mer brukervennlig en Telenor sin nettside. Fargevalget på telenor er litt roligere med mørkere farger som gjør vennligheten på øynene litt roligere, men One Call har til gjengjeld gjort mye som kan være avansert simplere.\\

Begge nettstedene er enkelt for alle målgruppene å nå. Det meste av informasjon er bygget opp slik at man finner det enkelt og veiviserene tar deg gjennom ting med mange valg.




\subsubsection*{Sammenligning}
Nettsidene skiller seg i hovedsak på fargevalg og profil valg. Telenor har valgt et mørkt design som gir siden et ekstra preg av proffesjonalitet. Hvor Onecall heller har valgt et litt mer ungt og hipt design. Telenor har også en avdeling for bedrift og Onecall ikke har dette. De har heller dette som et eget selvskap, Network Norway. Onecall bruker også mye  av sin nettside på å vise frem at de har norges mest fornøyde kunder og andre fordeler de mener de kan tilby iforhold til andre. Telenor holder seg her helt nøytrale.

\end{document}
